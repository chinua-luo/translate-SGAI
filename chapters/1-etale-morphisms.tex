\setcounter{chapter}{0}
\chapter{Étale morphisms}

To simplify the exposition we assume that all preschemes in the following are locally Noetherian (at least, starting from section 2).


\section{Basics of differential calculus}

Let $X$ be a prescheme on $Y$, and $\Delta_{X/Y}$ the diagonal morphism $X\to X\times_Y X$.
This is an immersion, and thus a closed immersion of $X$ into an open subset $V$ of $X\times_Y X$.
Let $\ideal_X$ be the ideal of the closed sub-prescheme corresponding to the diagonal in $V$ (N.B. if one really wishes to do things intrinsically, without assuming that $X$ is separated over $Y$ — a \unsure{misleading} hypothesis — then one should consider the set-theoretic inverse image of $\O_{X\times X}$ in $X$ and denote by $\ideal_X$ the augmentation ideal in the above\ldots).
The sheaf $\ideal_X/\ideal_X^2$ can be thought of as a quasi-coherent sheaf on $X$, which we denote by $\Omega_{X/Y}^1$.
It is of finite type if $X\to Y$ is of finite type.
It behaves well with respect to a change of base $Y'\to Y$.
We also introduce the sheaves $\O_{X\times_Y X}/\ideal_X^{n+1}=\mathscr{P}^n_{X/Y}$, which are sheaves of \emph{rings} on $X$, giving us preschemes denoted by $\Delta_{X/Y}^n$ and called the \emph{$n$-th infinitesimal neighbourhood of $X/Y$}.
The polysyllogism is entirely trivial, even if rather long\footnote{cf. EGA~IV~16.3.}; it seems wise to not discuss it until we use it for something helpful, with smooth morphisms.


\section{Quasi-finite morphisms}

\begin{prop}
    Let $A\to B$ be a local homomorphism (N.B. all rings are now Noetherian) and $\mathfrak{m}$ the maximal ideal of $A$.
    Then the following conditions are equivalent:
    \begin{enumerate}[\normalfont(i)]
        \item $B/\mathfrak{m}B$ is of finite dimension over $k=A/\mathfrak{m}$.
        \item $\mathfrak{m}B$ is \unsure{[\ldots]} and $B/\mathfrak{r}(B)=k(B)$ is an extension of $k=k(A)$.
        \item The completion $\hat{B}$ of $B$ is finite over the completion $\hat{A}$ of $A$.
    \end{enumerate}
\end{prop}

We then say that $B$ \emph{is quasi-finite} over $A$.
A morphism $f\colon X\to Y$ is said to be quasi-finite in $x$ (or the $Y$-prescheme $f$ is said to be quasi-finite in $x$) if $\O_x$ is quasi-finite over $\O_{f(x)}$.
\completelyunsure
A morphism is said to be quasi-finite if it is quasi-finite in each point\footnote{In EGA~II~6.2.3 we further suppose that $f$ is of finite type.}.

\begin{cor}
    If $A$ is complete then quasi-finiteness is equivalent to finiteness.
\end{cor}

We could give the usual polysyllogism (i), (ii), (iii), (iv), (v) for quasi-finite morphisms, but that doesn't seem necessary here.


\section{Unramified morphisms}

\begin{prop}
    Let $f\colon X\to Y$ be a morphism of finite type, $x\in X$, and $y=f(x)$.
    Then the following conditions are equivalent:
    \begin{enumerate}[\normalfont(i)]
        \item $\O_x/\mathfrak{m}_y\O_x$ is a finite separable extension of $k(y)$.
        \item $\Omega_{X/Y}^1$ is \unsure{null} in $x$.
        \item The diagonal morphism $\Delta_{X/Y}$ is an open immersion on a neighbourhood of $x$.
    \end{enumerate}
\end{prop}

For the implication (i) $\implies$ (ii), we are brought by Nakayama to the case where $Y=\Spec(k)$ and $X=\Spec(k')$, where it is well known and otherwise trivial by the definition of separable; (ii) $\implies$ (iii) comes from a nice and easy characterisation of open immersions, using Krull; (iii) $\implies$ (i) follows as well from reducing to the case where $Y=\Spec(k)$ and the diagonal morphism is everywhere an open immersion.
One must then prove that $X$ is finite \completelyunsure and this leads us to consider the case where $k$ is algebraically closed.
But then every closed point of $X$ is isolated (since it is identical to the inverse image of the diagonal by the morphism $X\to X\times_k X$ defined by $x$), whence $X$ is finite.
We can thus suppose that $X$ reduces to a point, of the ring $A$, and so $A\otimes_k A\to A$ is an isomorphism, hence $A=k$.\qed

\begin{defn}
    \begin{enumerate}[\normalfont(a)]
        \item We then say that $f$ is \emph{unramified} in $x$, or that $X$ is unramified in $x$ on $Y$.
        \item Let $A\to B$ be a local homomorphism.
            We say that it is \emph{unramified}, or that $B$ is a local \emph{unramified} algebra on $A$, if $B/mathfrak{r}(A)B$ is a finite separable extension of $A/\mathfrak{r}(A)$, i.e. if $\mathfrak{r}(A)B=\mathfrak{r}(B)$ and $k(B)$ is a separable extension of $k(A)$\footnote{Cf. regrets in III~1.2.}.
    \end{enumerate}
\end{defn}

\begin{rem}
    The fact that $B$ is unramified over $A$ \completelyunsure.
    Unramified implies quasi-finite.
\end{rem}

\begin{cor}
    The set of points where $f$ is unramified is open.
\end{cor}

\begin{cor}
    Let $X'$ and $X$ be two preschemes of finite type over $Y$, and $g\colon X'\to X$ a $Y$-morphism.
    If $X$ is unramified over $Y$ then the graph morphism $\Gamma_g\colon X'\to X\times_Y X$ is an open immersion.
\end{cor}

In effect, this is the inverse image of the diagonal morphism $X\to X\times_Y X$ by
\begin{equation*}
    g\times_Y \id_{X'}\colon X'\times_Y X\to X\times_Y X.
\end{equation*}
One can also introduce the annihilator ideal $\mathfrak{d}_{X/Y}$ of $\Omega_{X/Y}^1$, called the \unsure{different} ideal of $X/Y$; it defines a closed sub-prescheme of $X$ which, set theoretically, is the set of point where $X/Y$ is ramified, i.e. not unramified.

\begin{prop}
    \begin{enumerate}[\normalfont(i)]
        \item An immersion is ramified.
        \item The composition of two ramified morphisms is also ramified.
        \item Base extension of a ramified morphisms is also ramified.
    \end{enumerate}
\end{prop}

\unsure{We don't care so much about (ii) or (iii)} (the second seems more \unsure{entertaining} to me).
We can also \completelyunsure by giving a few punctual comments; it is no more general than in appearance (except in the case of definition b) and also boring.
We obtain, as per usual, the corollaries:

\begin{cor}
    \begin{enumerate}[\normalfont(i)]
        \setcounter{enumi}{3}
        \item The cartesian product of two unramified morphisms is unramified.
        \item If $gf$ is unramified then so too is $f$.
        \item If $f$ is unramified then so too is $f_\text{red}$.
    \end{enumerate}
\end{cor}

\begin{prop}
    Let $A\to B$ be a local homomorphism and suppose that the \unsure[residue?]{residual} extension $k(B)/k(A)$ is trivial with $k(A)$ algebraically closed.
    In order for $B/A$ to be unramified it is necessary and sufficient that $\hat{B}$ be (as an $\hat{A}$-algebra) a quotient of $\hat{A}$.
\end{prop}

\begin{rem}
    \begin{itemize}
        \item In the case where we don't suppose that the \unsure{residual} extension is trivial, we can bring ourselves to the case where it is by taking a suitable finite flat extension on $A$ which \unsure{destroys} the aforementioned extension.
        \item Consider the example where $A$ is the local ring of an ordinary double point of a curve, and $B$ a point of the \unsure{normalised curve}: then $A\subset B$, $B$ is unramified over $A$ with trivial \unsure{residual} extension, and $\hat{A}\to\hat{B}$ is surjective but \emph{not injective}.
        We are thus going to strengthen the notion of unramifiedness.
    \end{itemize}
\end{rem}


\section{Étale morphisms. Étale covers.}
